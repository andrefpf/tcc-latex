% !TEX jobname = proposta-tcc-andre
% !TEX options=--shell-escape
% !TEX output_directory = ./.temp/

\documentclass{ufsc-thesis-rn46-2019}

\usepackage[T1]{fontenc} % fontes
\usepackage[utf8]{inputenc} % UTF-8
\usepackage{lipsum} % Gerador de texto
\usepackage{pdfpages} % Inclui PDF externo (ficha catalográfica)
\usepackage{xcolor, colortbl, multirow} % Gerar tabelas
\usepackage{minted} % Usado para mostrar código
\usepackage{cco-ufsc-tcc-tables}

\definecolor{shadecolor}{HTML}{C0C0C0}
\newmintinline[mt]{latex}{fontsize=\normalsize}
\newmintinline[mft]{latex}{fontsize=\footnotesize}
\setminted{fontsize=\tiny,linenos,xleftmargin=2em}
\setmintedinline{breaklines,breakbytokenanywhere}

% 
\titulo[]{Método para Controle de Taxa Alvo na Compressão de Light Fields Seguindo o Padrão JPEG Pleno Parte 2 - 4DTM}
\autor{André Filipe da Silva Fernandes}
\data{\today}
\instituicao{Universidade Federal de Santa Catarina}
\centro{Centro Tecnológico}
\local{Florianópolis}
\tcc
\departamento{Departamento de Informática e Estatística}
\curso{Ciência da Computação}
\titulode{Bacharel em Ciência da Computação}

\instituicaoResponsavelConcordancia{ECL/INE/CTC}
\foneResponsavelConcordancia{37217511}
\areaDeConcentracao{\textit{Hardware}}

\orientador{Prof. Ismael Seidel, Dr.}
\membrabanca{Prof. Lorem Ipsum, Dr.}{Universidade Federal da Terra de Ninguém}
\membrobanca{Prof. Lorem Ipsum, Dr.}{Universidade Federal da Terra de Ninguém}
\membrobanca{Prof. Lorem Ipsum, Dr.}{Universidade Federal da Terra de Ninguém}
\coordenador{Prof. Lorem Ipsum, Dr}


\begin{document}

\pretextual%
\imprimircapa%
\imprimirfolhaderosto*
\imprimirfolhadeaprovacaodeproposta{}

% \protect\incluirfichacatalografica{ficha.pdf}
% \imprimirfolhadecertificacao

\begin{resumo}[Resumo]
    Em algoritmos de compressão de dados com perdas existe uma necessidade natural de balanceamento da taxa de compressão com a taxa de distorção,
    pois taxas de compressão muito elevadas produzem arquivos de má qualidade, enquanto taxas de compressão muito baixas proporcionam arquivos de
    tamanhos impraticáveis.
    A parte 2 do padrão JPEG Pleno, que é dedicada à codificação de lightfields, possui um modo de transformadas 4D que utiliza um multiplicador
    Lagrangiano para realizar este balanceamento \cite{4d_codec}. O problema de se utilizar o multiplicador Lagrangiano para este fim é a dificuldade
    em estimar o tamanho dos dados de saída, e portanto escolher os parâmetros de maneira adequada. Este trabalho propõe um algoritmo capaz de 
    otimizar a taxa de distorção baseando-se em uma taxa alvo de bits por píxel, e a implementação deste algoritmo na parte 2 do software de 
    referência do padrão JPEG Pleno para o modo de transformadas 4D.

    % Aqui deve ser inserido um resumo de 150 a 500 palavras (limitação de tamanho dada pela BU). A linguagem deve ser português e a hifenização já foi alterada. O resumo em português deve preceder o resumo em inglês, mesmo que o trabalho seja escrito em inglês. A BU também diz que deve ser usada a voz ativa e o discurso deve ser na 3ª pessoa. A estrutura do resumo pode ser similar a estrutura usada em artigos: Contexto -- Problema -- Estado da arte -- Solução proposta  -- Resultados.

    % Atenção! a BU exige separação através de ponto (.). Ela recomanda de 3 a 5 keywords
    \vspace{\baselineskip} 
    \textbf{Palavras-chave:} Lightfields. JPEG Pleno. Compressão de dados. Problema da taxa alvo.
\end{resumo}


% O * evita que apareça no sumário
\tableofcontents*


\textual

% Capítulos
\chapter{Introdução}
    Light fields são uma nova tecnologia de representação de imagens, capaz de retratar simultaneamente a intensidade e o ângulo de raios de luz 
    incidentes em cada ponto de um plano, superando assim, limitantes de tecnologias mais tradicionais como fotografias e vídeos. Deste modo,
    light fields proporcionam algumas funcionalidades novas perante estes paradigmas já estabelecido, como a capacidade de simular o efeito paralax
    através de tipos especiais de telas, permitindo uma experiência mais natural e imersiva do que no cinema 3D tradicional. Também é possível
    extrair profundidade de campo de lightfields com certa facilidade, realizar pequenas alterações de perspectiva e até alterar o foco da imagem em
    etapas de pós processamento \cite{lightfields_survey}.
    
    % não tenho certeza se minha interpretação dessa quadrupla está muito correta
    Assim como em outras modalidades de imagens, a representação de light fields em sua forma bruta exige uma quantidade impraticavelmente grande de
    memória para armazenamento, tornando desejável o uso de técnicas de compressão, especialmente devido à natureza quadridimensional destes dados
    que codificam cada ponto da imagem como uma quádrupla de coeficientes $(u, v, s, t)$ \cite{4d_codec}, representando respectivamente as coordenadas horizontais, 
    verticais, ângulo de incidência perpendicular ao eixo horizontal e ângulo de incidência perpendicular ao eixo vertical. Neste contexto foi criado
    o grupo JPEG Pleno com a intenção de estabelecer padrões para codificação e representação de light fields e de outras imagens plenopticas, como
    \textit{point clouds} e hologramas. Até o momento o padrão inclui dois modos para codificar light fields: 4D-Transform mode, que é baseado na transformada
    discreta de cosseno em 4 dimensões - \textit{4 dimensional discrete cossine transform} (4D-DCT); e 4D-Prediction mode que é um modelo preditivo, 
    capaz de gerar estágios intermediários de partes da imagem baseado no seu contexto.

    O modo de transformada 4D, no qual este trabalho se concentra, particiona a imagem em blocos quadridimensionais não sobrepostos,
    limitados por um tamanho máximo configurável. Com estes blocos criados, é possível dividí-los recursivamente em até 16 blocos menores, seguindo algum
    critério de taxa de distorção (preferencialmente concentrando mais blocos em partes mais detalhadas da imagem) de forma que o custo Lagrangiano seja
    minimizado, conforme definido na equação \ref{eqn:lagrangian_cost}, onde J é o custo Lagrangiano, D é o quadrado do erro em relação ao original e
    R é a taxa de codificação. A forma como estes blocos foram divididos é modelada como uma \textit{hexadecatree} e
    codificada em um bitplane, para possibilitar a reconstrução posterior da imagem, visto que eles passarão por uma 4D-DCT \cite{4d_codec}.
    
    \begin{equation}
        \label{eqn:lagrangian_cost}
        J = D + \lambda R
    \end{equation}

    O problema da taxa alvo, neste contexto, é a necessidade de fazer com que os dados de saída possuam um tamanho próximo a uma taxa préviamente
    especificada. Ter este controle do tamanho dos dados após comprimidos é uma característica útil para \textit{codecs}, especialmente se utilizados em
    aplicações de \textit{streaming}, por exemplo, onde é necessário controlar precisamente a vazão dos dados de acordo com a capacidade da rede. A utilização
    de um custo Lagrangiano para o particionamento dos blocos torna difícil a tarefa de estimar a taxa alvo através dos parâmetros de entrada, por isso
    o presente trabalho resolve este problema modificando o algoritmo de particionamento para que ele se limite a um tamanho fixo de saída que será 
    passado como parâmetro de entrada, dispensando o uso do operador Lagrangiano e facilitando a operação do codec para usuários ou outras aplicações.


% Seções
\section{Objetivos}
    \subsection{Objetivo Geral}
        O objetivo deste trabalho é implementar no software de referência do JPEG Pleno um algoritmo capaz de controlar a qualidade do arquivo de
        saída baseando-se em uma taxa alvo de bits.

    \subsection{Objetivos Específicos}
        \begin{itemize}
            \item Demonstrar que o algoritmo proposto atinge a taxa alvo conforme o esperado.
            \item Avaliar se os light fields gerados através da taxa alvo mantêm qualidade semelhante àqueles gerados através do multiplicador
            Lagrangiano.
            \item Avaliar a eficiência e desempenho do novo algoritmo em relação ao anterior.
        \end{itemize}

\section{Método de Pesquisa}
    No início do projeto será feito o estudo da base de código do software de referência do JPEG Pleno, bem como a leitura de trabalhos correlatos ao
    tema proposto, visando conhecer técnicas empregadas especialmente em codecs de imagem e vídeo e que podem ser utilizadas no desenvolvimento deste 
    trabalho. A partir do conhecimento adquirido através deste estudo, o algoritmo proposto será implementado no software de referência do JPEG Pleno
    (JPLM) \footnote{Disponível publicamente em: \url{https://gitlab.com/wg1/jpeg-pleno-refsw}.}. A linguagem utilizada será C++17. Assim que o algoritmo 
    estiver funcional serão feitas as comparações de qualidade e desempenho com o modelo original, conforme estabelecido pelas Common Test Conditions (CTC) \cite{CTC_JPEG_Pleno}
    definidas para o padrão. As comparações de qualidade serão feitas através da Relação Sinal-Ruído de Pico - \textit{Peak signal to noise ratio} (PSNR),
    o desempenho será medido em tempo de execução e a eficiência da codificação através de curvas de \textit{Rate Distortion} (RD).

\section{Cronograma}
    \resizebox{\textwidth}{!}{%
    \begin{tabular}{|l|llllllllll|}
    \hline
    \rowcolor[HTML]{C0C0C0} 
    \multicolumn{1}{|c|}{\cellcolor[HTML]{C0C0C0}} &
    \multicolumn{5}{c|}{\cellcolor[HTML]{C0C0C0}\textbf{2022}} &
    \multicolumn{5}{c|}{\cellcolor[HTML]{C0C0C0}\textbf{2023}} \\ \cline{2-11}
    \rowcolor[HTML]{C0C0C0} 
    \multicolumn{1}{|c|}{\multirow{-2}{*}{\cellcolor[HTML]{C0C0C0}\textbf{Etapas}}} &
    \multicolumn{1}{l|}{\cellcolor[HTML]{C0C0C0}\textbf{ago}} &
    \multicolumn{1}{l|}{\cellcolor[HTML]{C0C0C0}\textbf{set}} &
    \multicolumn{1}{l|}{\cellcolor[HTML]{C0C0C0}\textbf{out}} &
    \multicolumn{1}{l|}{\cellcolor[HTML]{C0C0C0}\textbf{nov}} &
    \multicolumn{1}{l|}{\cellcolor[HTML]{C0C0C0}\textbf{dez}} &
    \multicolumn{1}{l|}{\cellcolor[HTML]{C0C0C0}\textbf{jan}} &
    \multicolumn{1}{l|}{\cellcolor[HTML]{C0C0C0}\textbf{fev}} &
    \multicolumn{1}{l|}{\cellcolor[HTML]{C0C0C0}\textbf{mar}} &
    \multicolumn{1}{l|}{\cellcolor[HTML]{C0C0C0}\textbf{abr}} &
    \textbf{mai} \\ \hline
    Desenvolvimento da solução &
    \multicolumn{1}{l|}{\cellcolor[HTML]{C0C0C0}} &
    \multicolumn{1}{l|}{\cellcolor[HTML]{C0C0C0}} &
    \multicolumn{1}{l|}{\cellcolor[HTML]{C0C0C0}} &
    \multicolumn{1}{l|}{\cellcolor[HTML]{C0C0C0}} &
    \multicolumn{1}{l|}{\cellcolor[HTML]{C0C0C0}} &
    \multicolumn{1}{l|}{\cellcolor[HTML]{C0C0C0}} &
    \multicolumn{1}{l|}{\cellcolor[HTML]{C0C0C0}} &
    \multicolumn{1}{l|}{} &
    \multicolumn{1}{l|}{} &
    \\ \hline
    Relatório projeto I &
    \multicolumn{1}{l|}{} &
    \multicolumn{1}{l|}{} &
    \multicolumn{1}{l|}{} &
    \multicolumn{1}{l|}{\cellcolor[HTML]{C0C0C0}} &
    \multicolumn{1}{l|}{\cellcolor[HTML]{C0C0C0}} &
    \multicolumn{1}{l|}{} &
    \multicolumn{1}{l|}{} &
    \multicolumn{1}{l|}{} &
    \multicolumn{1}{l|}{} &
    \\ \hline
    Rascunho projeto II &
    \multicolumn{1}{l|}{} &
    \multicolumn{1}{l|}{} &
    \multicolumn{1}{l|}{} &
    \multicolumn{1}{l|}{} &
    \multicolumn{1}{l|}{} &
    \multicolumn{1}{l|}{} &
    \multicolumn{1}{l|}{} &
    \multicolumn{1}{l|}{\cellcolor[HTML]{C0C0C0}} &
    \multicolumn{1}{l|}{\cellcolor[HTML]{C0C0C0}} &
    \\ \hline
    Defesa &
    \multicolumn{1}{l|}{} &
    \multicolumn{1}{l|}{} &
    \multicolumn{1}{l|}{} &
    \multicolumn{1}{l|}{} &
    \multicolumn{1}{l|}{} &
    \multicolumn{1}{l|}{} &
    \multicolumn{1}{l|}{} &
    \multicolumn{1}{l|}{} &
    \multicolumn{1}{l|}{\cellcolor[HTML]{C0C0C0}} &
    \\ \hline
    Ajustes e envio Final &
    \multicolumn{1}{l|}{} &
    \multicolumn{1}{l|}{} &
    \multicolumn{1}{l|}{} &
    \multicolumn{1}{l|}{} &
    \multicolumn{1}{l|}{} &
    \multicolumn{1}{l|}{} &
    \multicolumn{1}{l|}{} &
    \multicolumn{1}{l|}{} &
    \multicolumn{1}{l|}{\cellcolor[HTML]{C0C0C0}} & \cellcolor[HTML]{C0C0C0}
    \\ \hline
    \end{tabular}%
    }
\section{Recursos Humanos}
    \resizebox{\textwidth}{!}{
    \begin{tabular}{|l|l|}
    \hline
    \rowcolor[HTML]{C0C0C0} 
    \multicolumn{1}{|c|}{
    \cellcolor[HTML]{C0C0C0}\makebox[8em]{}\textbf{Nome}\makebox[8em]{}} & \makebox[8em]{}\textbf{Função}\makebox[8em]{} \\ 
    \hline
    André Fernandes & Autor\\ 
    \hline
    Ismael Seidel &  Orientador\\
    \hline
    Renato Cislaghi & Coordenador\\
    \hline
    a definir & Membro da banca I\\ 
    \hline
    a definir & Membro da banca II\\ 
    \hline
    \end{tabular}
    }
\section{Comunicação}
    \resizebox{\textwidth}{!}{
    \begin{tabular}{|l|l|l|l|l|}
    \hline
    \rowcolor[HTML]{C0C0C0} 
    \multicolumn{1}{|c|}{\cellcolor[HTML]{C0C0C0}\textbf{\begin{tabular}[c]{@{}c@{}}O que precisa ser \\ comunicado\end{tabular}}} & \textbf{Por quem} & \textbf{Para quem} & \textbf{\begin{tabular}[c]{@{}l@{}}Melhor forma de \\ comunicação\end{tabular}} & \textbf{\begin{tabular}[c]{@{}l@{}}Quando ou com que \\ frequencia\end{tabular}} \\ \hline
    Entrega da proposta de TCC & Autor & Coordenador & Sistema de TCC & única vez\\
    \hline 
    Entrega do relatório I & Autor & Coordenador & Sistema de TCC & única vez \\ 
    \hline 
    Entrega do relatório II & Autor & Coordenador &  Sistema de TCC & única vez \\
    \hline 
    Reuniões com o orientador & Autor & Orientador & Pessoalmente/videochamada & quando necessário\\ 
    \hline
    \end{tabular}
    }
\section{Riscos}

% Please add the following required packages to your document preamble:
% \usepackage[table,xcdraw]{xcolor}
% If you use beamer only pass "xcolor=table" option, i.e. \documentclass[xcolor=table]{beamer}

% \begin{table}[]
% \centering
\resizebox{\textwidth}{!}{%
\begin{tabular}{|l|l|l|l|l|l|}
  \hline
  \rowcolor[HTML]{C0C0C0} 
  \multicolumn{1}{|c|}{\cellcolor[HTML]{C0C0C0}\textbf{Risco}} &
    \multicolumn{1}{c|}{\cellcolor[HTML]{C0C0C0}\textbf{Probabilidade}} &
    \multicolumn{1}{c|}{\cellcolor[HTML]{C0C0C0}\textbf{Impacto}} &
    \multicolumn{1}{c|}{\cellcolor[HTML]{C0C0C0}\textbf{Prioridade}} &
    \multicolumn{1}{c|}{\cellcolor[HTML]{C0C0C0}\textbf{Estratégia de resposta}} &
    \multicolumn{1}{c|}{\cellcolor[HTML]{C0C0C0}\textbf{Ações de Prevenção}} \\ \hline
  \rowcolor[HTML]{FFFFFF}

  \textbf{Perda de dados} &
    baixa &
    alto &
    alta &
    Utilizar ferramenta online de versionamento (GitHub) &
    Geração de Backups \\ \hline

  \textbf{Problemas de Saúde} &
    baixa &
    alto &
    alta &
    Realizar Tratamento &
    Manter bons hábitos e ser prudente \\ \hline
  
  \textbf{Alteração no Tema} &
    média &
    alto &
    alta &
    Buscar novo tema ou modificar escopo atual &
    Manter interação constante com o orientador \\ \hline
  
  \textbf{Alteração no Cronograma} &
    média &
    alto &
    alta &
    Realizar as adaptações necessárias &
    Manter-se dentro do cronograma \\ \hline
  
  \textbf{Expiração da Licença das ferramentas de síntese e simulação} &
    baixa &
    médio &
    alta &
    Renovar a licença ou substituir a ferramenta de simulação &
    Manter-se dentro do cronograma \\ \hline

  \rowcolor[HTML]{FFFFFF} 
  Alteração do cronograma &
    baixa &
    médio &
    média &
    redefinição do croonograma &
    não se aplica \\ \hline
\end{tabular}%
}


% 

\postextual

\nocite{jplm_code}
\bibliography{references}

\anexos
\imprimirdeclaracaodeconcordanciatcc{}
\end{document}