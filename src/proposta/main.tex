% !TEX jobname = proposta-tcc-andre
% !TEX options=--shell-escape
% !TEX output_directory = ./.temp/

\documentclass[embeddedlogo]{ufsc-thesis-rn46-2019}

\usepackage[T1]{fontenc} % fontes
\usepackage[utf8]{inputenc} % UTF-8
\usepackage{lipsum} % Gerador de texto
\usepackage{pdfpages} % Inclui PDF externo (ficha catalográfica)

% Usado para mostrar código
\usepackage{minted}
\newmintinline[mt]{latex}{fontsize=\normalsize}
\newmintinline[mft]{latex}{fontsize=\footnotesize}
\setminted{fontsize=\tiny,linenos,xleftmargin=2em}
\setmintedinline{breaklines,breakbytokenanywhere}

\titulo[]{Solução para o problema da taxa alvo durante a codificação de light fields}

\autor{André Filipe da Silva Fernandes}
\data{\today}
\instituicao{Universidade Federal de Santa Catarina}
\centro{Centro Tecnológico}
\local{Florianópolis}
\tcc
\departamento{Departamento de Informática e Estatística}
\curso{Ciência da Computação}
\titulode{Bacharel em Ciência da Computação}

\orientador{Prof. Ismael Seidel, Dr.}
\coorientador{Prof. José Luís A. Güntzel (?), Dr.}
\membrabanca{Prof. Lorem Ipsum, Dr.}{Universidade Federal de Santa Catarina}
\membrobanca{Prof. Lorem Ipsum, Dr.}{Universidade Federal de Santa Catarina}
\membrobanca{Prof. Lorem Ipsum, Dr.}{Universidade Federal da Terra de Ninguém}
\coordenador{Prof. Lorem Ipsum, Dr}


\begin{document}

\pretextual%
\imprimircapa%
\imprimirfolhaderosto*
\imprimirfolhadeaprovacaodeproposta{}

% \protect\incluirfichacatalografica{ficha.pdf}
% \imprimirfolhadecertificacao

\begin{resumo}[Resumo]
    Em algoritmos de compressão de dados com perdas existe uma necessidade natural de balanceamento da taxa de compressão com a taxa de distorção,
    pois taxas de compressão muito elevadas produzem arquivos de má qualidade, enquanto taxas de compressão muito baixas proporcionam arquivos de
    tamanhos impraticáveis.
    A parte 2 do padrão JPEG Pleno, que é dedicada à codificação de lightfields, possui um modo de transformadas 4D que utiliza um multiplicador
    Lagrangiano para realizar este balanceamento \cite{4d_codec}. O problema de se utilizar o multiplicador Lagrangiano para este fim é a dificuldade
    em estimar o tamanho dos dados de saída, e portanto escolher os parâmetros de maneira adequada. Este trabalho propõe um algoritmo capaz de 
    otimizar a taxa de distorção baseando-se em uma taxa alvo de bits por píxel, e a implementação deste algoritmo na parte 2 do software de 
    referência do padrão JPEG Pleno para o modo de transformadas 4D.

    % Aqui deve ser inserido um resumo de 150 a 500 palavras (limitação de tamanho dada pela BU). A linguagem deve ser português e a hifenização já foi alterada. O resumo em português deve preceder o resumo em inglês, mesmo que o trabalho seja escrito em inglês. A BU também diz que deve ser usada a voz ativa e o discurso deve ser na 3ª pessoa. A estrutura do resumo pode ser similar a estrutura usada em artigos: Contexto -- Problema -- Estado da arte -- Solução proposta  -- Resultados.

    % Atenção! a BU exige separação através de ponto (.). Ela recomanda de 3 a 5 keywords
    \vspace{\baselineskip} 
    \textbf{Palavras-chave:} Lightfields. JPEG Pleno. Compressão de dados. Problema da taxa alvo.
\end{resumo}


% O * evita que apareça no sumário
\tableofcontents*


\textual%

\chapter{Introdução}
O que é um lightfield?
Por que comprimir?
Como funciona mais ou menos a codificação? \cite{lightfields_survey}
O que eu pretendo fazer?

\section{Objetivos}
    \subsection{Objetivo Geral}
        O objetivo deste trabalho é implementar no software de referência do JPEG Pleno um algoritmo capaz de otimizar 
        a taxa de distorção baseando-se em uma taxa alvo de bits por píxel.
    
    \subsection{Objetivos Específicos}
        \begin{itemize}
            \item Demonstrar que o algoritmo proposto atinge a taxa alvo conforme o esperado.
            \item Avaliar se os lightfields gerados através da taxa alvo mantêm qualidade semelhante àqueles gerados através do multiplicador Lagrangiano.
            \item Avaliar a eficiência e desempenho do novo algoritmo em relação ao anterior.
        \end{itemize}

\section{Método de Pesquisa}
    No início do projeto será feito o estudo da base de código do software de referência do JPEG Pleno, bem como a leitura de trabalhos correlatos ao tema proposto, 
    visando conhecer técnicas empregadas especialmente em codecs de imagem e vídeo e que podem ser utilizadas no desenvolvimento deste trabalho.
    A partir do conhecimento adquirido através deste estudo, o algoritmo será desenvolvido a partir do software de referência, utilizando a linguagem c++.
    Assim que o algoritmo estiver funcional serão feitas as comparações de qualidade e desempenho com o modelo original, utilizando os lightfields fornecidos pelo
    dataset de lightfields do JPEG Pleno com parâmetros variados. As comparações de qualidade serão feitas através da Relação Sinal-Ruído de Pico - 
    \textit{Peak signal to noise ratio} (PSNR), enquanto o desempenho será medido em tempo de execução.

\section{Cronograma}

\section{Custos}

\section{Recursos Humanos}

\section{Comunicação}

\section{Riscos}

\postextual

\nocite{jplm_code}
\bibliography{references}


\end{document}