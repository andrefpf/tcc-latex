\section{Método de Pesquisa}
    No início do projeto será feito o estudo da base de código do software de referência do JPEG Pleno, bem como a leitura de trabalhos correlatos ao
    tema proposto, visando conhecer técnicas empregadas especialmente em codecs de imagem e vídeo e que podem ser utilizadas no desenvolvimento deste 
    trabalho. A partir do conhecimento adquirido através deste estudo, o algoritmo será desenvolvido a partir do software de referência, utilizando a 
    linguagem c++. Assim que o algoritmo estiver funcional serão feitas as comparações de qualidade e desempenho com o modelo original, utilizando os 
    lightfields fornecidos pelo dataset de light fields do JPEG Pleno com parâmetros variados. As comparações de qualidade serão feitas através da 
    Relação Sinal-Ruído de Pico - \textit{Peak signal to noise ratio} (PSNR), enquanto o desempenho será medido em tempo de execução.
