\section{Método de Pesquisa}
    No início do projeto será feito o estudo da base de código do software de referência do JPEG Pleno, bem como a leitura de trabalhos correlatos ao
    tema proposto, visando conhecer técnicas empregadas especialmente em codecs de imagem e vídeo e que podem ser utilizadas no desenvolvimento deste 
    trabalho. A partir do conhecimento adquirido através deste estudo, o algoritmo proposto será implementado no software de referência do JPEG Pleno
    (JPLM) \footnote{Disponível publicamente em: \url{https://gitlab.com/wg1/jpeg-pleno-refsw}.}. A linguagem utilizada será C++17. Assim que o algoritmo 
    estiver funcional serão feitas as comparações de qualidade e desempenho com o modelo original, conforme estabelecido pelas Common Test Conditions (CTC) \cite{CTC_JPEG_Pleno}
    definidas para o padrão. As comparações de qualidade serão feitas através da Relação Sinal-Ruído de Pico - \textit{Peak signal to noise ratio} (PSNR),
    o desempenho será medido em tempo de execução e a eficiência da codificação através de curvas de \textit{Rate Distortion} (RD).
