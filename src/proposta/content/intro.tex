\chapter{Introdução}
O que é um lightfield?
Por que comprimir?
Como funciona mais ou menos a codificação? \cite{lightfields_survey}
O que eu pretendo fazer?

\section{Objetivos}
    \subsection{Objetivo Geral}
        O objetivo deste trabalho é implementar no software de referência do JPEG Pleno um algoritmo capaz de otimizar 
        a taxa de distorção baseando-se em uma taxa alvo de bits por píxel.
    
    \subsection{Objetivos Específicos}
        \begin{itemize}
            \item Demonstrar que o algoritmo proposto atinge a taxa alvo conforme o esperado.
            \item Avaliar se os lightfields gerados através da taxa alvo mantêm qualidade semelhante àqueles gerados através do multiplicador
            Lagrangiano.
            \item Avaliar a eficiência e desempenho do novo algoritmo em relação ao anterior.
        \end{itemize}

\section{Método de Pesquisa}
    No início do projeto será feito o estudo da base de código do software de referência do JPEG Pleno, bem como a leitura de trabalhos correlatos ao
    tema proposto, visando conhecer técnicas empregadas especialmente em codecs de imagem e vídeo e que podem ser utilizadas no desenvolvimento deste 
    trabalho. A partir do conhecimento adquirido através deste estudo, o algoritmo será desenvolvido a partir do software de referência, utilizando a 
    linguagem c++. Assim que o algoritmo estiver funcional serão feitas as comparações de qualidade e desempenho com o modelo original, utilizando os 
    lightfields fornecidos pelo dataset de lightfields do JPEG Pleno com parâmetros variados. As comparações de qualidade serão feitas através da 
    Relação Sinal-Ruído de Pico - \textit{Peak signal to noise ratio} (PSNR), enquanto o desempenho será medido em tempo de execução.

\section{Cronograma}

\section{Custos}

\section{Recursos Humanos}

\section{Comunicação}

\section{Riscos}