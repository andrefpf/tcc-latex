\chapter{Introdução}
Light fields são uma modalidade de representação de imagens e vídeos que tenta expandir a capacidade representativa destes meios através da captura de
múltiplos ângulos simultaneamente. Através disso, os light fields proporcionam funcionalidades novas perante o paradigma já estabelecido, como simular
o efeito de paralax através de tipos especiais de telas, permitindo uma experiência mais natural e imersiva do que no cinema 3D tradicional. Também é
possível transformar light fields em imagens tradicionais com a vantagem de poder realizar pequenas modificações de ângulo ou ajustar o foco nas 
etapas de pós processamento. Light fields podem ser capturados através de um agrupamento de câmeras, ou através de arrays de microlentes.

Assim como em outras modalidades de imagens, light fields em sua forma bruta exigem uma quantidade impraticavelmente grande de memória para 
armazenamento, especialmente devido à natureza quadridimensional destes dados que codificam tanto a posição quanto o ângulo incidente em cada ponto
da imagem, portanto a utilização de técnicas de compressão são desejáveis. Neste contexto foi criado o grupo JPEG Pleno com a intenção de estabelecer
padrões para codificação e representação de light fields e de outras imagens plenopticas, como point clouds e hologramas. Até o momento o padrão 
inclui dois modos para codificar light fields: 4D-Transform mode, que é baseado na transformada discreta de cosseno em 4 dimensões - 
\textit{4 dimensional discrete cossine transform} (4D-DCT); e 4D-Prediction mode que é um modelo preditivo, capaz de gerar estágios intermediários de
partes da imagem baseado no seu contexto.

O modo de transformada 4D, no qual este trabalho se concentra, particiona a imagem em blocos quadridimensionais não sobrepostos,
limitados por um tamanho máximo configurável. Com estes blocos criados, é possível dividí-los recursivamente em até 16 blocos menores, seguindo algum
critério de taxa de distorção (preferencialmente concentrando mais blocos em partes mais detalhadas da imagem) de forma que o custo Lagrangiano seja
minimizado, conforme definido na equação \ref{eqn:lagrangian_cost}. A forma como estes blocos foram divididos é modelada como uma hexadecatree e
codificada em um bitplane, para possibilitar a reconstrução posterior da imagem, visto que eles passarão por uma 4D-DCT.

Ter um controle do tamanho dos dados após comprimidos é uma característica útil para codecs, especialmente se utilizado em aplicações de streaming, 
por exemplo, onde é necessário controlar precisamente a vazão dos dados de acordo com a capacidade da rede. A utilização de um custo Lagrangiano
para o particionamento dos blocos torna difícil a tarefa de estimar a taxa alvo através dos parâmetros de entrada. O presente trabalho 
resolve este problema modificando o algoritmo de particionamento para que ele se limite a um tamanho fixo de saída que será passado como parâmetro
de entrada, dispensando o uso do operador Lagrangiano.


\begin{equation}
    \label{eqn:lagrangian_cost}
    J = D + \lambda R
\end{equation}

% O que é um lightfield?
% Por que comprimir?
% Como funciona mais ou menos a codificação? \cite{lightfields_survey}
% O que eu pretendo fazer?

\section{Objetivos}
    \subsection{Objetivo Geral}
        O objetivo deste trabalho é implementar no software de referência do JPEG Pleno um algoritmo capaz de otimizar 
        a taxa de distorção baseando-se em uma taxa alvo de bits por píxel.
    
    \subsection{Objetivos Específicos}
        \begin{itemize}
            \item Demonstrar que o algoritmo proposto atinge a taxa alvo conforme o esperado.
            \item Avaliar se os light fields gerados através da taxa alvo mantêm qualidade semelhante àqueles gerados através do multiplicador
            Lagrangiano.
            \item Avaliar a eficiência e desempenho do novo algoritmo em relação ao anterior.
        \end{itemize}

\section{Método de Pesquisa}
    No início do projeto será feito o estudo da base de código do software de referência do JPEG Pleno, bem como a leitura de trabalhos correlatos ao
    tema proposto, visando conhecer técnicas empregadas especialmente em codecs de imagem e vídeo e que podem ser utilizadas no desenvolvimento deste 
    trabalho. A partir do conhecimento adquirido através deste estudo, o algoritmo será desenvolvido a partir do software de referência, utilizando a 
    linguagem c++. Assim que o algoritmo estiver funcional serão feitas as comparações de qualidade e desempenho com o modelo original, utilizando os 
    lightfields fornecidos pelo dataset de light fields do JPEG Pleno com parâmetros variados. As comparações de qualidade serão feitas através da 
    Relação Sinal-Ruído de Pico - \textit{Peak signal to noise ratio} (PSNR), enquanto o desempenho será medido em tempo de execução.

\section{Cronograma}

\section{Custos}

\section{Recursos Humanos}

\section{Comunicação}

\section{Riscos}